\documentclass[11pt]{article}
\usepackage{amsmath,amssymb,graphicx}
\usepackage[margin=1in]{geometry}

\title{Charged Elastic Rod in 3D: Discrete Energy, Gradient, and BFGS Optimization}
\author{Numerical Methods Programming Assignment (Starter Note)}
\date{}

\begin{document}
\maketitle

\section{Overview}
We model a closed filament in $\mathbb{R}^3$ (think: a charged DNA loop in solution) as $N$ nodes
\[
X = (\mathbf{x}_0,\dots,\mathbf{x}_{N-1}),\qquad \mathbf{x}_i\in\mathbb{R}^3,
\]
with \emph{periodic indexing} $\mathbf{x}_{i+N}=\mathbf{x}_i$. The goal is to minimize a total energy
\[
E(X) = E_{\text{bend}}(X) + E_{\text{stretch}}(X) + E_{\text{coul}}(X),
\]
using a quasi-Newton method (BFGS) in $\mathbb{R}^{3N}$.

The optimization is \emph{nonconvex} and exhibits multiple local minima. Good numerical behavior requires:
(i) correct analytic gradients, (ii) a line search, and (iii) robust updates.

\section{Discrete energy}
\subsection{Bending (curvature penalty)}
A standard discrete curvature surrogate is the second difference:
\[
\mathbf{b}_i = \mathbf{x}_{i+1} - 2\mathbf{x}_i + \mathbf{x}_{i-1}.
\]
We define
\[
E_{\text{bend}} = k_b \sum_{i=0}^{N-1} \|\mathbf{b}_i\|^2.
\]
Large $k_b$ favors smoother curves.

\subsection{Stretching (near-inextensibility)}
Let segment vectors be $\mathbf{s}_i = \mathbf{x}_{i+1}-\mathbf{x}_i$ with length $r_i=\|\mathbf{s}_i\|$.
We penalize deviation from rest length $\ell_0$:
\[
E_{\text{stretch}} = k_s \sum_{i=0}^{N-1} (r_i-\ell_0)^2.
\]
Typically $k_s\gg k_b$ to keep the filament close to inextensible.

\subsection{Screened Coulomb repulsion}
Charges on the filament repel. In ionic solution this is often modeled by Debye--H\"uckel screening:
\[
U(r)=\frac{q^2 e^{-\kappa r}}{r},
\]
with screening parameter $\kappa\ge 0$ (set $\kappa=0$ for unscreened Coulomb). The total repulsion is
\[
E_{\text{coul}} = \sum_{0\le i<j\le N-1\atop (i,j)\ \text{not neighbors}} U(\|\mathbf{x}_i-\mathbf{x}_j\|).
\]
We exclude adjacent neighbors (including the wrap neighbor $(0,N-1)$) to avoid double-counting interactions
already represented by stretching/bending.

\section{Gradients (high level)}
Your optimizer will require $\nabla E(X)\in\mathbb{R}^{3N}$. The starter repository provides a C++ routine
that returns both $E$ and $\nabla E$.

Two typical patterns:
\begin{itemize}
\item For bending, each $\mathbf{x}_i$ appears in $\mathbf{b}_{i-1},\mathbf{b}_i,\mathbf{b}_{i+1}$, yielding a local stencil.
\item For stretching and Coulomb terms, derivatives follow from the chain rule:
\[
\frac{\partial}{\partial \mathbf{x}} \|\mathbf{v}\| = \frac{\mathbf{v}}{\|\mathbf{v}\|},
\]
and for Coulomb:
\[
\frac{d}{dr}\left(\frac{q^2 e^{-\kappa r}}{r}\right)
= q^2 e^{-\kappa r}\left(-\frac{1}{r^2} - \frac{\kappa}{r}\right).
\]
\end{itemize}


\section{Self-avoidance via segment--segment WCA interaction}

Point--point repulsion is insufficient to prevent self-intersection of a discretized filament.
Instead, we model \emph{self-avoidance} via short-range repulsion between \emph{segments} of the polyline.

Let the $i$-th segment be
\[
\mathbf{p}_i(u) = \mathbf{x}_i + u(\mathbf{x}_{i+1}-\mathbf{x}_i), \qquad u\in[0,1],
\]
and similarly
\[
\mathbf{p}_j(v) = \mathbf{x}_j + v(\mathbf{x}_{j+1}-\mathbf{x}_j), \qquad v\in[0,1].
\]

Define the squared distance function
\[
\phi(u,v) = \|\mathbf{p}_i(u) - \mathbf{p}_j(v)\|^2.
\]
The segment--segment distance is
\[
d_{ij} = \min_{u,v\in[0,1]} \|\mathbf{p}_i(u)-\mathbf{p}_j(v)\| = \sqrt{\phi(u^\*,v^\*)},
\]
where $(u^\*,v^\*)$ satisfies the first-order conditions
\[
\partial_u\phi(u,v)=0,\qquad \partial_v\phi(u,v)=0,
\]
together with the box constraints $u,v\in[0,1]$.

\subsection{Weeks--Chandler--Andersen (WCA) potential}
We use the repulsive part of the Lennard--Jones potential:
\[
U(d)=
\begin{cases}
4\varepsilon\Bigl[(\sigma/d)^{12}-(\sigma/d)^6\Bigr]+\varepsilon,
& d < 2^{1/6}\sigma,\\
0, & d\ge 2^{1/6}\sigma.
\end{cases}
\]
Define the self-avoidance energy
\[
E_{\mathrm{sa}}=\sum_{(i,j)\in\mathcal{P}} U(d_{ij}),
\]
where $\mathcal{P}$ ranges over \emph{non-adjacent} segment pairs (exclude $(i,i+1)$ with $(i\pm 1,i\pm 2)$ and the wrap neighbors).

\subsection{Gradient requirement}
You must compute $\nabla E_{\mathrm{sa}}$ with respect to all node positions $\mathbf{x}_k$.
This requires differentiating through the closest-point problem: the minimizers $(u^\*,v^\*)$ depend implicitly
on the segment endpoints $(\mathbf{x}_i,\mathbf{x}_{i+1},\mathbf{x}_j,\mathbf{x}_{j+1})$.

\section{Confinement potential (to induce supercoiling-like states)}
To obtain nontrivial equilibria (coiling/packing) with purely repulsive self-avoidance, we add a weak confining potential
\[
E_{\mathrm{conf}} = k_c \sum_{i=0}^{N-1} \|\mathbf{x}_i\|^2,
\]
with $k_c>0$ small. This encourages the filament to pack into a bounded region while self-avoidance prevents self-intersection.

\section{BFGS task}
You will implement BFGS in Python to minimize $f(x)$ where $x\in\mathbb{R}^{3N}$ is the flattened coordinate vector.
At iteration $k$:
\begin{enumerate}
\item compute direction $p_k = -H_k \nabla f(x_k)$ where $H_k$ approximates the inverse Hessian;
\item choose step length $\alpha_k$ using a line search (Armijo or Wolfe conditions);
\item update $x_{k+1}=x_k+\alpha_k p_k$;
\item update $H_k$ using the BFGS inverse-Hessian update with curvature check $y_k^Ts_k>0$.
\end{enumerate}

\section{Suggested experiments}
\begin{itemize}
\item Compare BFGS vs. gradient descent on the same initialization.
\item Vary $(k_b,k_s,\kappa)$ to see the qualitative change in equilibria.
\item Observe sensitivity to the initial curve (multiple local minima).
\end{itemize}

\end{document}
